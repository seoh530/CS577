% with answers
% \documentclass[solutionorbox,answers]{exam}
% no answers
\documentclass[solutionorbox]{exam}

%%%%%%%%%%%%%%%%%%%%%%%%%%%%%%%%%%%%%%%%%%%%%%%%%%%%%%%%%%%%%%%
% Update to change header
\newcommand{\courseName}{CS 577}
\newcommand{\assignmentName}{Assignment 3: Greedy Algorithms}
\newcommand{\semester}{Fall 2023}
%%%%%%%%%%%%%%%%%%%%%%%%%%%%%%%%%%%%%%%%%%%%%%%%%%%%%%%%%%%%%%%

\usepackage[utf8]{inputenc}
\usepackage[T1]{fontenc}

\usepackage{amsmath}
\usepackage{amsfonts}
\usepackage{amsthm}
\usepackage{booktabs}

\usepackage[ruled]{algorithm2e}
\usepackage{graphicx}

\usepackage{hyperref}

\pagestyle{headandfoot}
\runningheadrule
\firstpageheader{\courseName}{\huge \assignmentName}{\semester}
\runningheader{\courseName}
{\assignmentName}
{\semester}
\firstpagefooter{}{}{}
\runningfooter{}{Page \thepage\ of \numpages}{}

\begin{document}

\begin{center}
\fbox{\parbox{5.5in}{\centering
Answer the questions in the boxes provided on the
question sheets. If you run out of room for an answer,
add a page to the end of the document. \\
\vspace{0.1in}
}}
\end{center}
\vspace{0.1in}
\makebox[0.48\textwidth]{Name:\enspace\hrulefill} \qquad
\makebox[0.48\textwidth]{Wisc id:\enspace\hrulefill}

\begin{questions}

\section*{Greedy Algorithms}

\question
In one or two sentences, describe what a greedy algorithm is. Your definition should be informal, something you could share with a non computer scientist.
\nopagebreak
\begin{solutionbox}{1in}
    % answer here
\end{solutionbox}

\question
There are many different problems all described as ``scheduling'' problems. In the following questions, pay attention to the details of the problem setup, as they will change each time!

\begin{parts}
  \part
  Let each job have a start time, an end time, and a value. We want to schedule as much value of non-conflicting jobs as possible. Use a counterexample to show that Earliest Finish First (the greedy algorithm we used for jobs with all equal value) does NOT work in this case.
  
  \begin{solutionbox}{1in}
    % answer here
  \end{solutionbox}
  
  \part
  \emph{Kleinberg, Jon. Algorithm Design (p. 191, q. 7)} Now let each job consist of two durations. A job $i$ must be preprocessed for $p_i$ time on a supercomputer, and then finished for $f_i$ time on a standard PC. There are enough PCs available to run all jobs at the same time, but there is only one supercomputer (which can only run a single job at a time). The completion time of a schedule is defined as the earliest time when all jobs are done running on both the supercomputer and the PCs. Give a polynomial time algorithm that finds a schedule with the earliest completion time possible.
  \begin{solutionbox}{\stretch{1}}
    % answer here
  \end{solutionbox}
  %%%%%%%%%%%%%%%%%%%%%%%%
  \pagebreak
  %%%%%%%%%%%%%%%%%%%%%%%%
  \part
  Prove the correctness and efficiency of your algorithm from part (b).
\nopagebreak
  \begin{solutionbox}{\stretch{1}}
    % answer here
  \end{solutionbox}
\end{parts}

\pagebreak
\question
 \emph{Kleinberg, Jon. Algorithm Design (p. 190, q. 5)}
 \begin{parts}
   \part
Consider a long straight road with houses scattered along it. We want to place cell phone towers along the road so that every house is within four miles of at least one tower. Give an efficient algorithm that achieves this goal using the minimum possible number of towers.
\begin{solutionbox}{\stretch{1}}
    % answer here
\end{solutionbox}

\part
Prove the correctness of your algorithm.
\begin{solutionbox}{\stretch{3}}
    % answer here
\end{solutionbox}
 \end{parts}
 
 \pagebreak
 \question
 \emph{Kleinberg, Jon. Algorithm Design (p. 197, q. 18)} Your friends are planning to drive north from Madison to the town of Superior, Wisconsin over winter break. They have drawn a directed graph with nodes representing potential stops and edges representing the roads between them.
 
 They have also found a weather forecasting site that can accurately predict how long it will take to traverse one of the edges on their graph, given the starting time $t$. This is important because some of the roads on their graph are affected strongly by the seasons and by extreme weather. It's guaranteed that it never takes negative time to traverse an edge, and that you can never arrive earlier by starting later.
 
\begin{parts}
\part
 Design an algorithm your friends can use to plot the quickest route. You may assume that they start at time $t=0$, and that the predictions made by the weather forecasting site are accurate.
  \begin{solutionbox}{\stretch{1}}
    % answer here
 \end{solutionbox}

 \pagebreak
 
 \part
 Demonstrate how your algorithm works using a small example with 6 nodes. Your demonstration should include any data structures you maintain during the execution of your algorithm and any queries you make to the weather forecasting site. For example, if your algorithm maintains a ``current path'' that grows from (M)adison to (S)uperior, you might show something like the following table:
 \quad
 \begin{tabular}{c|c}
Path      & Total time \\
\hline
M      & 0 \\
M,A & 2 \\
M,A,E & 5 \\
M,A,E,F & 6 \\
M,A,E & 5 \\
M,A,E,H & 10 \\
M,A,E,H,S & 13
 \end{tabular}
   \begin{solutionbox}{\stretch{1}}
    % answer here
 \end{solutionbox}

 \end{parts}
 
\pagebreak
\section*{Coding Question}
\question
Implement the optimal algorithm for interval scheduling (for a definition of the problem, see the Greedy slides on Canvas) in either C, C++, C\#, Java, or Python. Be efficient and implement it in $O(n \log n)$ time, where $n$ is the number of jobs.

The input will start with an positive integer, giving the number of instances that follow. For each instance, there will be a positive integer, giving the number of jobs. For each job, there will be a pair of positive integers $i$ and $j$, where $i < j$, and $i$ is the start time, and $j$ is the end time.

A sample input is the following:
\begin{verbatim}
2
1
1 4
3
1 2
3 4
2 6
\end{verbatim}
The sample input has two instances. The first instance has one job to schedule with a start time of 1 and an end time of 4. The second instance has 3 jobs.

For each instance, your program should output the number of intervals scheduled on a separate line. Each output line should be terminated by a newline. The correct output to the sample input would be:
\begin{verbatim}
1
2
\end{verbatim}

\end{questions}

\end{document}
